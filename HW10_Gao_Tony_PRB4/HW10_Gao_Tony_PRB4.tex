\documentclass[letter,10pt]{article}
\usepackage[letterpaper, lmargin=.5in, rmargin=.5in, tmargin=.5in]{geometry}
\usepackage[utf8]{inputenc}
\usepackage{verbatim}
\usepackage{color}
\usepackage[]{amsmath}
\usepackage{mathtools}
\usepackage{pdfpages}
\usepackage{enumitem}
\usepackage{listings}
\usepackage{amssymb}
\usepackage{amsfonts}
\usepackage{graphicx}
\usepackage{subfig}
% \usepackage{pstool}
\usepackage{algorithm}
\usepackage{array}
\usepackage{tikz}
\usetikzlibrary{matrix}
\usepackage[binary-units]{siunitx}
\usepackage[backref=page,
            pageanchor=true,
            plainpages=false,
            pdfpagelabels,
            bookmarks,
            bookmarksnumbered,
            linkbordercolor={1 0.5 0.5},
            citebordercolor={0 0.5 0},
]{hyperref}

\usepackage{array,mathtools}
\newcommand*{\carry}[1][1]{\overset{#1}}
\newcolumntype{B}[1]{r*{#1}{@{\,}r}}

\newcommand{\eg}{\textit{e.g.},~}
\newcommand{\etc}{\textit{etc.}}
\let\eqref\undefined

\newcommand{\figref}[1]{Figure~\ref{fig:#1}}
\newcommand{\algoref}[1]{Algorithm~\ref{algo:#1}}
\newcommand{\secref}[1]{Section~\ref{sec:#1}}
\newcommand{\tabref}[1]{Table~\ref{tab:#1}}
\newcommand{\eqref}[1]{Equation~\ref{eq:#1}}
\newcommand{\figlabel}[1]{\label{fig:#1}}
\newcommand{\algolabel}[1]{\label{algo:#1}}
\newcommand{\seclabel}[1]{\label{sec:#1}}
\newcommand{\tablabel}[1]{\label{tab:#1}}
\newcommand{\eqlabel}[1]{\label{eq:#1}}

\DeclarePairedDelimiter\abs{\lvert}{\rvert}%
\DeclarePairedDelimiter\norm{\lVert}{\rVert}%


% Title Page
\title{CMPSCI 453\\HW10, Problem Set 4}
\author{Tony Gao}

\begin{document}
\maketitle

%\begin{abstract}
%\end{abstract}

\section{Problem 1}

\begin{enumerate}[label=\alph*]
	\item False, each sequence number in the sender window in SR must be acknowledged for the sender window to move forward. It is possible for the receiver to acknowledge a sequence number outside of the receiver window, but impossible for the sender to receive an ack outside of the sender window. The sender window does not move forward until the earliest unacked packet is acked.
	
	\item False, when the sender window moves forward in GBN, it is guaranteed that all packets up to and including send\_base have been acknowledged. If a certain packet is lost, all packets that are transferred after it within the sender window will be retransmitted until all packets have been cumulatively acked. It is possible that the receiver receives a packet that it has already acked. However, the sender is guaranteed by design to not send a packet that is outside of the sender window.
	
	\item True. The behavior in RDT3.0 is the same as SR with a sender/receiver window of 1. If a corrupted packet is received by the receiver, no ack is sent, causing a timeout and retransmission. If a duplicate packet is received by the receiver, the sequence number of the duplicate packet is acked. If an ack is lost, the sender will timeout and resend the packet. The sender does not move to the next sequence number until the previous sequence number is correctly acked. This mirrors SR.
	
	\item False. As a counterexample: If a duplicate packet is received by the receiver in GBN, the sequence number of the next expected sequence number is sent due to the cumulative ack. In RDT 3.0, the sequence number of the duplicate packet is sent in the ack. Additionally, there is no receiver window in GBN, so the premise of this statement is false.
	
\end{enumerate}

\section{Problem 2}

For the time intervals, interval notation is used to denote open and closed intervals.

\begin{enumerate} 
	\item 
	\begin{itemize}
		\item Slow Start: [1,4), [8, 11), [27, 31)
		\item Congestion Avoidance: [4,8), [11,16), [17, 27) [31, 40]
		\item Fast recovery: [16, 17) \\
		 Fast recovery has not timed out because cwnd continues to grow after T=17, implying the timeout is about 1 RTT. At T = 17, a new ack for a sequence coming after the duplicate ack number must have arrived at the sender, transitioning TCP to congestion avoidance per the FSM.
	\end{itemize}
		
	\item 
	\begin{itemize}
		\item Packet is lost just after T = 7. Timeout occurs just before T=8 since cwnd = 1 at T=8. 
		\item Packet is lost just after T=15, cwnd is set to $12 / 2 + 3 = 9$ at T=16 due to triple duplicate ack loss event.
		\item Packet is lost just after T=26. Timeout occurs at T=27 since cwnd is set to 1 at T=27. 
	\end{itemize}
	
	
	\item 
	\begin{itemize}
		\item SSThresh set to $11/2 = 5$ at T=8 since cwnd = 11 at T=7, when the timeout loss event occurred
		\item SSThresh set to $12 / 2 = 6$ at T= 16 since cwnd = 12 at T=15, when the triple duplicate ack loss event occurred.
		\item SSThresh set to $19 / 2 = 9$ at T=27 since cwnd = 19 at T=26, when the timeout loss event occurred.
	\end{itemize}
	
\end{enumerate}

\section{Problem 3}
For the time intervals, interval notation is used to denote open and closed intervals.
	
\begin{enumerate}
	\item 
	\begin{itemize}
		\item Slow start: [1,13), [20,22), [35,38) 
		\item Congestion Avoidance: [13,20), [22, 27), [28,32), [33,35), [38,40]
		\item Fast Recovery: [27, 28), [32,33)
	\end{itemize}		
			
	\item 
	\begin{itemize}
		\item Lost just after T=3, timeout just before T=4.
		\item Lost just after T=7, timeout just before T=8
		\item Lost just after T=11, timeout just before T=12
		\item Lost just after T=19, timeout just before T=20
		\item Lost just after T=26, triple duplicate ack just before T=27
		\item Lost just after T=31, triple duplicate ack just before T=32
		\item Lost just after T=34, timeout just before T=35
	\end{itemize}

	\item 
	\begin{itemize}
		\item ssthresh set to $4/2 = 2$ at T=4.
		\item ssthresh set to $4/2 = 2$ at T=8.
		\item ssthresh set to $4/2 = 2$ at T=12.
		\item ssthresh set to $8/2 = 4$ at T=20.
		\item ssthresh set to $8/2 = 4$ at T=27.
		\item ssthresh set to $11/2 = 5$ at T=32.
		\item ssthresh set to $10/2 = 5$ at T=35.
	\end{itemize}
\end{enumerate}

\section{Problem 4}

\begin{table}[h!]
	\centering
	\begin{tabular}{|l|l|l|l|l|}
		\hline
		S-to-R & Time Segment Sent & S-to-R sequence \# & Time Segment Received & R-to-S ACK \# \\ \hline
		Segment 1          & 1                 & 91                            & 8                     & $91 + 564$               \\ \hline
		Segment 2          & 2                 & $91+564$                      & 9                     & $91 + 2 * 564$           \\ \hline
		Segment 3          & 3                 & $91+2*564$                    & segment was lost     & segment was lost, no ack \\ \hline
		Segment 4          & 4                 & $91+3*564$                    & segment was lost     & segment was lost, no ack \\ \hline
		Segment 5          & 5                 & $91+4*564$                    & 12                    & $91 + 2 * 564$           \\ \hline
	\end{tabular}
\end{table}

\begin{table} [h!]
	\centering
	\begin{tabular}{| l | l | l | p{8cm}|}
		\hline
		S-to-R & Time Segment Sent & S-to-R sequence \# & Explanation                                                                                                                                                                                                                 \\ \hline
		Segment 1          & 15                & $91+5*564$                    & Currently 4 unacked segments in the cwnd since ack for segment sent at T=1 was received at T=15, so there is space for 1 more sequence to be sent                                                                           \\ \hline
		Segment 2          & 16                & $91+6*564$                    & Currently 4 unacked segments in cwnd since ack for segment sent at T=2 was received at T=16, so there is space for 1 more sequence to be sent                                                                               \\ \hline
		Segment 4          & 19                & $91+2*564$                    & The cwnd was full from T=16 until T=19. There have not yet been 3 duplicate acks, so this transmission must be due to timeout on the oldest not acked segment, which is the one sent at T=3 with sequence number $91+2*564$ \\ \hline
	\end{tabular}
\end{table}

\section{Problem 5}

\begin{enumerate}[label=\alph*.]
	\item With 4 byte sequence numbers, the maximum amount of sequence numbers and thus maximum number of bytes is $2^{32} - 1 B = 4294967295B$
	
	\item 
	\begin{flalign}
	\text{Number of segments that the file must be divided into}&:  ceil((2 ^ {32} - 1) / 536) = 8012999 &\\
	\text{Total header overhead}&: 8012999 * 66B = 528857934B &\\
	\text{Total transmission size}&:  4294967295B + 528857934B = 4823825229B & \\
	\text{Transmission time to nearest ms}&: 4823825229B / (155Mbps / 8 * 1000000)Bps \approx 248.972s &
	\end{flalign}
	
\end{enumerate}

\section{Problem 6}

\begin{table}[h!]
	\centering
	\begin{tabular}{|l|l|l|l|l|}
		\hline
		S-to-R & Time Segment Sent & S-to-R sequence \# & Time Segment Received & R-to-S ACK \# \\ \hline
		Segment 1          & 1                 & 148                            & 8                     & $148 + 509$               \\ \hline
		Segment 2          & 2                 & $148+509$                      & 9                     & $148 + 2 * 509$           \\ \hline
		Segment 3          & 3                 & $148 + 2* 509$                 & Not received          & Not received, no ack sent \\ \hline
		Segment 4          & 4                 & $148 + 3 * 509$                & Not received          & Not received, no ack sent \\ \hline
		Segment 5          & 5                 & $148 + 4 * 509$                & Not received          & Not received, no ack sent \\ \hline
	\end{tabular}
\end{table}

\begin{enumerate}
	\item Segment 1 R-to-S explanation: TCP Cumulative ACK defines an ack for a received segment as equal to the sequence number of the next expected byte, which is $ 148 + 509 $
	\item Segment 2 R-to-S explanation: TCP Cumulative ACK defines an ack for a received segment as equal to the sequence number of the next expected byte, which is $ 148 + 2 * 509 $
	\item Segment 3 R-to-S explanation: There is no ack sent, because the segment was never received.
	\item Segment 4 R-to-S explanation: There is no ack sent, because the segment was never received.
	\item Segment 5 R-to-S explanation: There is no ack sent, because the segment was never received.
\end{enumerate}

\end{document}
